\documentclass{article}
\usepackage[utf8]{inputenc}
\usepackage{mathtools}
\usepackage{amsfonts}
\usepackage{geometry}
\usepackage{minted}
\geometry{margin=1in}

\title{Maratona}
\author{}
\date{}

\begin{document}

\maketitle

\textbf{IMPORTANT: 1-INDEXED DATA STRUCTURES, MUST SWAP MIN FOR MAX,
CHANGE INF IF NECESSARY (DEFAULT $2^{30}$), DEFINE INT LONG LONG IF NECESSARY}

\tableofcontents
\newpage

\section{Python}
\subsection{Fenwick Tree}
\inputminted{python}{code/fenwick.py}

\section{C++}
\subsection{Fenwick Tree}
\inputminted[obeytabs=true,tabsize=4]{cpp}{code/fenwick.cpp}
\subsection{Fraction}
Mostly used for comparisons in geometry (Convex hull)
\inputminted[obeytabs=true,tabsize=4]{cpp}{code/fraction.cpp}
\subsection{2D Point}
\inputminted[obeytabs=true,tabsize=4]{cpp}{code/point.cpp}
\subsection{Dijkstra}
\inputminted[obeytabs=true,tabsize=4]{cpp}{code/dijkstra.cpp}
\subsection{RMQ}
\inputminted[obeytabs=true,tabsize=4]{cpp}{code/rmq.cpp}
\subsection{Sieve}
\inputminted[obeytabs=true,tabsize=4]{cpp}{code/sieve.cpp}
\subsection{Fast Exponentiation}
\inputminted[obeytabs=true,tabsize=4]{cpp}{code/fast_exp.cpp}
\subsection{Dinic}
\inputminted[obeytabs=true,tabsize=4]{cpp}{code/dinic.cpp}
\subsection{Persistent Segment Tree} 
Update at point and query of interval after update t
\inputminted[obeytabs=true,tabsize=4]{cpp}{code/persistentseg.cpp}
\subsection{Segment Tree with Lazy Propagation} 
\inputminted[obeytabs=true,tabsize=4]{cpp}{code/lazyseg.cpp}
\end{document}
